\begin{titlepage}
        \newgeometry{margin=0pt}
        \thispagestyle{empty}
        \begin{tikzpicture}[remember picture, overlay]
            % --- 背景 ---
            \coordinate (split top) at ($(current page.north west)!0.7!(current page.north east)$);
            \shade[left color=yancovercolor!80!black, right color=yancovercolor!80!white, shading angle=90] 
                (split top) rectangle (current page.south east);
            
            % --- 右侧装饰诗句 (保留原样作为装饰,不再暴露配置,如需修改可在此处硬改) ---
            \node[anchor=center] at ($(split top)!0.5!(current page.south east)$) {
                \ifx\kaishu\undefined\newCJKfontfamily\kaishu{KaiTi}\fi
                \fontsize{24}{40}\selectfont \kaishu \color{white}
                \begin{tabular}{@{}c@{\hspace{1em}}c@{}}
                    一 & 月 \\ 只 & 亮 \\ 猫 & 想 \\ 吃 & 着 \\ 了 & 我 \\ 我 & 的 \\ 的 & 心 \\ 奶 & 事 \\ 酪 &
                \end{tabular}
            };
            
            % --- 左侧白底面板 ---
            \node[fill=white, anchor=north west, minimum height=\paperheight, minimum width=0.7\paperwidth, inner sep=0pt] 
                at (current page.north west) (leftpanel) {};
            
            % --- 左侧文字内容 ---
            \node[anchor=north west, inner sep=2cm, text width=0.55\paperwidth] at (leftpanel.north west) {
                \vspace{3cm}

                % === 标题区域 ===
                % 主标题
                {\fontsize{42}{50}\selectfont \sffamily \bfseries \textcolor{black!85}{\@title}}
                \\[0.2em] % 这里控制主标题和副标题的距离,0.2em 表示很近
                {\fontsize{16}{24}\selectfont \sffamily \textcolor{gray}{\@subtitle}}
                \\[2.5cm] % 这里控制副标题和下方装饰线的距离

                % 装饰线
                {\color{amzchaptercolor}\rule{2cm}{4pt}}\\[2.5cm]
                
                % 固定摘要 (可在此处修改固定文案)
                % {\sffamily \large \color{black!70} \setstretch{1.4}
                % \qquad{}本模板专注于\textbf{简洁}与\textbf{优雅}。
                % 灵感来源于现代界面设计,它拥有充足的留白、
                % 清晰的排版以及结构化的布局。
                % \par}\vspace{3cm}
                
                {\sffamily \large \color{black!70} \setstretch{1.4}
                \csname @coverabstract\endcsname
                \par}\vspace{3cm}




                % 底部信息表格 (使用你设置的变量)
                \noindent
                \begin{tabular}{@{} p{0.45\linewidth} p{0.45\linewidth} @{}}
                    {\sffamily \bfseries \footnotesize \textcolor{gray}{作者}} & {\sffamily \bfseries \footnotesize \textcolor{gray}{发布日期}} \\[0.5em]
                    {\sffamily \large \textbf{\@author}} & {\sffamily \large \textbf{\@date}} \\[0.8em]
                    
                    {\sffamily \bfseries \footnotesize \textcolor{gray}{机构}} & {\sffamily \bfseries \footnotesize \textcolor{gray}{版本}} \\[0.5em]
                    {\sffamily \small \textcolor{black!80}{\@institute}} & {\sffamily \small \textcolor{black!80}{\@version}} \\[0.1em]
                    {\sffamily \small \textcolor{black!40}{\@instituteen,\@location}} & {} 
                \end{tabular}
            };
            
            % --- Logo ---
            \node[anchor=north east, xshift=-2cm, yshift=-2cm] at (leftpanel.north east) {
                \begin{tikzpicture}
                    \fill[amzchaptercolor!20] (0,0) circle (0.8cm);
                    \fill[amzchaptercolor] (0.4,0.4) circle (0.3cm);
                \end{tikzpicture}
            };
        \end{tikzpicture}
        \restoregeometry
    \end{titlepage}