\begin{titlepage}
    \newgeometry{margin=0pt}
    \thispagestyle{empty}
    
    \begin{tikzpicture}[remember picture, overlay]
        % 1. 右侧面板:自定义颜色背景,贯穿整个页面,无阴影
        % 计算分割点,假设左侧占 70%
        \coordinate (split top) at ($(current page.north west)!0.7!(current page.north east)$);
        \coordinate (split bottom) at ($(current page.south west)!0.7!(current page.south east)$);
        
        % 填充右侧区域
        \fill[yancovercolor] (split top) rectangle (current page.south east);
        
        % 2. 右侧内容:竖排文字,白色
        \node[anchor=center] at ($(split top)!0.5!(current page.south east)$) {
            % 尝试使用楷体,如果未定义则定义
            \ifx\kaishu\undefined
                \newCJKfontfamily\kaishu{KaiTi}
            \fi
            \fontsize{24}{40}\selectfont \kaishu \color{white}
            % 使用表格模拟竖排,从右向左读
            \begin{tabular}{@{}c@{\hspace{1em}}c@{}}
                一 & 月 \\
                只 & 亮 \\
                猫 & 想 \\
                吃 & 着 \\
                了 & 我 \\
                我 & 的 \\
                的 & 心 \\
                奶 & 事 \\
                酪 &    \\
            \end{tabular}
        };
        
        % 3. 左侧面板:白色背景,贯穿整个页面,有阴影
        \node[
            fill=white,
            anchor=north west,
            minimum height=\paperheight,
            minimum width=0.7\paperwidth,
            inner sep=0pt,
            % 阴影投向右侧
            % drop shadow={opacity=0.2, shadow xshift=10pt, shadow yshift=0pt}
        ] at (current page.north west) (leftpanel) {};
        
        % 4. 左侧内容
        \node[anchor=north west, inner sep=2cm, text width=0.55\paperwidth] at (leftpanel.north west) {
            
            % 顶部留白
            \vspace{3cm}
            
            % 标题
            {\fontsize{42}{50}\selectfont \sffamily \bfseries \textcolor{black!85}{Yan Book}}\\[0.2em]
            {\fontsize{42}{50}\selectfont \sffamily \bfseries \textcolor{black!85}{模板}}\\[1em]
            
            % 副标题
            {\fontsize{16}{24}\selectfont \sffamily \textcolor{gray}{一个现代化的 \LaTeX\ 书籍模板}}\\[2.5cm]
            
            % 装饰线
            {\color{amzchaptercolor}\rule{2cm}{4pt}}\\[2.5cm]
            
            % 摘要
            {\sffamily \large \color{black!70} \setstretch{1.4}
            本模板专注于\textbf{简洁}与\textbf{优雅}。
            灵感来源于现代界面设计,它拥有充足的留白、
            清晰的排版以及结构化的布局,非常适合用于技术书籍、
            论文和学术笔记。
            \par}\vspace{3cm}
            
            % 底部信息
            \begin{minipage}{0.45\textwidth}
                {\sffamily \bfseries \footnotesize \textcolor{gray}{作者}}\\[0.5em]
                {\sffamily \large \textbf{yaan}}\\[0.2em]
                {\sffamily \small \textcolor{black!60}{安徽大学}}\\[0.1em]
                {\sffamily \tiny \textcolor{black!40}{Anhui University}}
            \end{minipage}%
            \begin{minipage}{0.45\textwidth}
                {\sffamily \bfseries \footnotesize \textcolor{gray}{发布日期}}\\[0.5em]
                {\sffamily \large \textbf{\today}}\\[0.2em]
                {\sffamily \small \textcolor{black!60}{版本 1.0}}
            \end{minipage}
        };

        % 5. 装饰元素:左侧面板右上角的极简几何图形
        \node[anchor=north east, xshift=-2cm, yshift=-2cm] at (leftpanel.north east) {
            \begin{tikzpicture}
                \fill[amzchaptercolor!20] (0,0) circle (0.8cm);
                \fill[amzchaptercolor] (0.4,0.4) circle (0.3cm);
            \end{tikzpicture}
        };
        
    \end{tikzpicture}
    \restoregeometry
\end{titlepage}