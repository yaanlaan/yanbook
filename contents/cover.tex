\begin{titlepage}
    \newgeometry{margin=0pt}
    \thispagestyle{empty}
    
    \begin{tikzpicture}[remember picture, overlay]
        % 1. 右侧面板:渐变色背景
        \coordinate (split top) at ($(current page.north west)!0.7!(current page.north east)$);
        \coordinate (split bottom) at ($(current page.south west)!0.7!(current page.south east)$);
        
        \shade[
            left color=yancovercolor!80!black, 
            right color=yancovercolor!80!white,
            shading angle=90
        ] (split top) rectangle (current page.south east);
        
        % 2. 右侧内容:竖排文字
        \node[anchor=center] at ($(split top)!0.5!(current page.south east)$) {
            \ifx\kaishu\undefined
                \newCJKfontfamily\kaishu{KaiTi}
            \fi
            \fontsize{24}{40}\selectfont \kaishu \color{white}
            \begin{tabular}{@{}c@{\hspace{1em}}c@{}}
                一 & 月 \\
                只 & 亮 \\
                猫 & 想 \\
                吃 & 着 \\
                了 & 我 \\
                我 & 的 \\
                的 & 心 \\
                奶 & 事 \\
                酪 &    \\
            \end{tabular}
        };
        
        % 3. 左侧面板背景
        \node[
            fill=white,
            anchor=north west,
            minimum height=\paperheight,
            minimum width=0.7\paperwidth,
            inner sep=0pt,
        ] at (current page.north west) (leftpanel) {};
        
        % 4. 左侧内容
        \node[anchor=north west, inner sep=2cm, text width=0.55\paperwidth] at (leftpanel.north west) {
            
            \vspace{3cm}
            
            % 标题
            {\fontsize{42}{50}\selectfont \sffamily \bfseries \textcolor{black!85}{Yan Book}}\\[0.2em]
            {\fontsize{42}{50}\selectfont \sffamily \bfseries \textcolor{black!85}{模板}}\\[1em]
            
            % 副标题
            {\fontsize{16}{24}\selectfont \sffamily \textcolor{gray}{一个现代化的 \LaTeX\ 书籍模板}}\\[2.5cm]
            
            % 装饰线
            {\color{amzchaptercolor}\rule{2cm}{4pt}}\\[2.5cm]
            
            % 摘要
            {\sffamily \large \color{black!70} \setstretch{1.4}
            \qquad{}本模板专注于\textbf{简洁}与\textbf{优雅}。
            灵感来源于现代界面设计,它拥有充足的留白、
            清晰的排版以及结构化的布局,非常适合用于技术书籍、
            论文和学术笔记。
            \par}\vspace{3cm}
            
            % ==========================================
            % 修改部分:底部信息改为表格对齐
            % ==========================================
            \noindent
            \begin{tabular}{@{} p{0.45\linewidth} p{0.45\linewidth} @{}}
                {\sffamily \bfseries \footnotesize \textcolor{gray}{作者}} & 
                {\sffamily \bfseries \footnotesize \textcolor{gray}{发布日期}} \\[0.8em]
                
                {\sffamily \large \textbf{yaan}} & 
                {\sffamily \large \textbf{\today}} \\[0.3em]
                
                {\sffamily \small \textcolor{black!60}{安徽大学}} & 
                {\sffamily \small \textcolor{black!60}{版本 1.0}} \\[0.1em]
                
                {\sffamily \tiny \textcolor{black!40}{Anhui University}} & 
                {} 
            \end{tabular}
            % ==========================================
        };

        % 5. 装饰元素
        \node[anchor=north east, xshift=-2cm, yshift=-2cm] at (leftpanel.north east) {
            \begin{tikzpicture}
                \fill[amzchaptercolor!20] (0,0) circle (0.8cm);
                \fill[amzchaptercolor] (0.4,0.4) circle (0.3cm);
            \end{tikzpicture}
        };
        
    \end{tikzpicture}
    \restoregeometry
\end{titlepage}