\section{文本盒子}

\epigraph{A mathematician is a machine for turning coffee into theorems.}{Paul Erdős}

模板提供了几种预定义的彩色盒子,用于不同类型的内容。

\subsection{定义与定理}

\begin{lstlisting}[language=TeX, frame=single]
\begin{dfnbox}[标题]{标签}
    内容...
\end{dfnbox}

\begin{thmbox}[标题]{标签}
    内容...
\end{thmbox}
\end{lstlisting}
也可以使用
\begin{lstlisting}[language=TeX, frame=single]
\begin{boxname*}{标题}
    有标题无记号
\end{boxname*}

\begin{boxname}{标题}
    无标题有记号
\end{boxname}

\end{lstlisting}
可用盒子:
\begin{itemize}
    \item \texttt{dfnbox}: 定义 (蓝色)
    \item \texttt{thmbox}: 定理 (紫色,需要 \texttt{math} 选项)
    \item \texttt{lembox}: 引理 (紫色,需要 \texttt{math} 选项)
    \item \texttt{exbox}: 示例 (黄色)
    \item \texttt{tecbox}: 技巧 (橙色)
    \item \texttt{genbox}: 通用信息 (绿色)
    \item \texttt{notebox}: 注意/警告 (红色,简单风格)
\end{itemize}

\section{代码块}

\subsection{Mac 风格窗口}
这些块模仿 macOS 窗口的外观。它们使用 \texttt{listings} 宏包,不需要 Python。

\textbf{浅色模式:}
\begin{lstlisting}[language=TeX, frame=single]
\begin{maccode}{python}
def hello():
    print("Hello")
\end{maccode}
\end{lstlisting}

\textbf{深色模式:}
\begin{lstlisting}[language=TeX, frame=single]
\begin{macdarkcode}{python}
def hello():
    print("Hello")
\end{macdarkcode}
\end{lstlisting}

\section{对照环境}

使用 \texttt{comparison} 环境并排显示两列内容,中间有垂直分隔线。

\begin{lstlisting}[language=TeX, frame=single]
\begin{comparison}
    左侧内容
    \tcblower
    右侧内容
\end{comparison}
\end{lstlisting}

\section{特殊功能}

\subsection{章节名言}
在章节开头添加名言:
\begin{lstlisting}[language=TeX, frame=single]
\epigraph{名言内容}{作者}
\end{lstlisting}

\subsection{术语表}
为术语表定义符号:
\begin{lstlisting}[language=TeX, frame=single]
\nomenclature{$x$}{x 的描述}
\end{lstlisting}
列表使用 \texttt{\textbackslash printnomenclature} 打印。

\subsection{全页背景}
在当前页面插入全页背景图片:
\begin{lstlisting}[language=TeX, frame=single]
\fullpagebg[0.5]{path/to/image.jpg}
\end{lstlisting}
可选参数控制不透明度 (默认为 1)。
