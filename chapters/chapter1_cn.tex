\epigraph{Mathematics is the queen of the sciences.}{Carl Friedrich Gauss}

\section{欢迎}

这是一个使用 \texttt{yanbook} 文档类的示例文档。

\begin{dfnbox}[Yanbook]{yanbook}
    \dfntxt{Yanbook} 是一个基于 \texttt{amznotes} 的自定义 LaTeX 文档类。
\end{dfnbox}

\begin{thmbox}[示理定理]{sample}
    这是一个定理盒子。
    \[ E = mc^2 \]
\end{thmbox}



\section{中文空格处理}

在 LaTeX 中处理中文时,空格的处理方式可能会让人困惑。以下是几种在中文中添加空格的方法:

\subsection{基本空格处理}

在中文文本中,直接输入空格通常不会显示,因为 CJK 字符(中文、日文、韩文)的排版规则与西文不同。例如:

\begin{verbatim}
这是没有空格的中文句子。
\end{verbatim}

\subsection{使用特殊命令}

要插入空格,您可以使用以下方法:

\begin{enumerate}
    \item 使用 \verb|\ | 命令:这\ 里\ 有\ 空\ 格
    \item 使用 \verb|~| 命令:这里~有~不间断空格
    \item 使用 \verb|\hspace{}| 命令:这里\hspace{1cm}有指定宽度的空格
    \item 使用 \verb|\qquad| 或 \verb|\quad|:这里\quad 有\qquad 更大的空格
\end{enumerate}

\subsection{中英文混排的空格}

在中英文混排时,LaTeX 通常会自动处理空格:

\begin{verbatim}
这是一个包含 English 和 数字 123 的句子。
\end{verbatim}

\subsection{调整中文字符间距}

如果需要调整中文字符之间的间距,可以临时使用:

\begin{verbatim}
\begin{CJKspace}
这 里 每 个 字 之 间 会 有 空 格
\end{CJKspace}
\end{verbatim}

但请注意,这不是标准的中文排版方式,一般情况下不推荐使用。

\subsection{实用技巧}

\begin{itemize}
    \item 在需要强调分隔的地方,使用标点符号而非空格
    \item 使用引号、括号等标点符号来组织文本结构
    \item 在中英文混排时,适当在英文前后添加空格以提高可读性
\end{itemize}